\documentclass[11pt,french]{beamer}

\usepackage{pgf}
\usetheme{Antibes}
\usecolortheme{dolphin} 
\usepackage{textcomp}
\usepackage[utf8]{inputenc} % pour les accents
\usepackage[T1]{fontenc} % pour les accents
\usepackage{ marvosym }
\begin{document}


\pgfdeclareimage[width=10cm]{logos_criugm}{logos_criugm}
\title{Bootstrap analysis of stable clusters in resting-state fMRI}
\author[P. Bellec]{\pgfimage[width=55pt]{basc_logo_large}\\Pierre Bellec \\ \tiny{\texttt{pierre.bellec@criugm.qc.ca}}\vspace{-0.4cm}}
\institute[NIAK]{NeuroImaging Analysis Kit \vspace{-0.3cm}}
\date[2015]{\pgfuseimage{logos_criugm}}

\frame{\titlepage}

\section{Introduction}

\frame{
\frametitle{Resting-state fMRI: functional connectivity map}
\begin{figure}
\pgfimage[width=0.8\linewidth]{fig_fcmri}\\
\end{figure}
\tiny{The posterior cingulate cortex is used as a seed to derive an individual resting-state functional connectivity map, identifying the default-mode network.}
}

\frame{
\frametitle{Resting-state fMRI: functional connectome}
\begin{figure}
\pgfimage[width=\linewidth]{fig_corr_map}\\
\end{figure}
}

\frame{
\frametitle{Resting-state fMRI: resting-state networks}
\begin{figure}
\pgfimage[width=\linewidth]{fig_rsn_connectome}\\
\end{figure}
}

\section{Hierarchical clustering on resting-state networks}

\subsection{Clustering}

\frame{
\frametitle{Clustering : unsupervised classification}
\begin{figure}
\pgfimage[width=\linewidth]{fig_clustering}\\
\end{figure}
\tiny{On the left, coordinates of individuals define their similarities; on the right, HC proceeds by iterative mergings. Many clustering algorithms exist, e.g. k-means, fuzzy k-means, spectral clustering, SOM, neural gas. See Jain, Pattern Recognition Letters, 2009, for a review.}\\
}

\frame{
\frametitle{Clustering : bi-scale approach in fMRI functional connectivity}
\begin{figure}
\pgfimage[width=\linewidth]{fig_lsni}\\
\end{figure}
\tiny{LSNI algorithm, see Bellec et al., 2006, Neuroimage, 29: 1231– 1243.}
}

\frame{
\frametitle{Adjacency matrix representation of a clustering}
\begin{figure}
\pgfimage[width=0.8\linewidth]{fig_partition}\\
\end{figure}
}

\subsection{Stability analysis}

\frame{
\frametitle{Individual-level bootstrap stability analysis}
\begin{figure}
\pgfimage[width=0.9\linewidth]{fig_stab_ind}\\
\end{figure}
}

\frame{
\frametitle{Consensus clustering}
\begin{figure}
\pgfimage[width=0.8\linewidth]{fig_consensus_clustering}\\
\end{figure}
\tiny{From Dansereau et al., Front. Neurosci. 8:419. doi: 10.3389/fnins.2014.00419.}
}

\frame{
\frametitle{Group-level stability analysis}
\begin{figure}
\pgfimage[width=0.9\linewidth]{fig_stab_group}\\
\end{figure}
\tiny{From Bellec et al., Neuroimage 2010.}
}

\frame{
\frametitle{Clustering : stability maps}
\begin{figure}
\pgfimage[width=\linewidth]{fig_stab_maps}\\
\end{figure}
}

\section{Multiresolution stepwise selection}

\frame{
\frametitle{Local maxima of stability}
\begin{figure}
\pgfimage[width=0.75\linewidth]{fig_msteps_part1}\\
\end{figure}
Individual stability contrasts for 43 subjects.\\
}

\frame{
\frametitle{Interpolation of stability matrices I}
\begin{figure}
\pgfimage[width=0.65\linewidth]{fig_interpolate_stability_part1}\\
\end{figure}
}

\frame{
\frametitle{Interpolation of stability matrices II}
\begin{figure}
\pgfimage[width=\linewidth]{fig_interpolate_stability_part2}\\
\end{figure}
\tiny{Bellec, Proceedings of the 2013 International Workshop on Pattern Recognition in Neuroimaging (PRNI), IEEE, 2013, pp. 54-57.}
}

\frame{
\frametitle{Multiresolution stepwise selection (MSTEPS) I}

\begin{beamerboxesrounded}{Forward MSTEPS procedure}
\begin{enumerate}
 \item Initialization: no resolution is selected.
 \item Select a resolution that has not yet been selected, with probability proportional to the residual sum of squares at this resolution.
 \item Iterate (2-3) until a predefined percentage of residual sum of squares across all resolutions is reached.
 \item Iterate the model selection $B$ times, and keep the model with smallest residual sum of squares.
\end{enumerate}
\end{beamerboxesrounded}

}

\frame{
\frametitle{Multiresolution stepwise selection (MSTEPS) II}
\begin{beamerboxesrounded}{Component-wise MSTEPS procedure}
\begin{enumerate}
 \item Initialization: run a forward MSTEPS.
 \item For each resolution of the model, try to replace it by any of the resolutions not currently in the model.
 \item Keep the model with the minimal residual sum of squares across all resolutions.
 \item Iterate (2-3) until it is not possible anymore to reduce the residual sum of squares.
\end{enumerate}
\end{beamerboxesrounded}
}

\frame{
\frametitle{Reproducibility of resolution selection}
\begin{figure}
\pgfimage[width=\linewidth]{fig_msteps_part2}\\
\end{figure}
}

\section{Examples of partitions}

\frame{
\frametitle{Group consensus clusters as a function of resolution}
\begin{figure}
\pgfimage[width=\linewidth]{fig_multiscale}\\
\end{figure}
\tiny{Bellec PRNI 2013.}
}


\frame{
\frametitle{Group consensus clusters \MVAt(resolution 2)}
\begin{figure}
\pgfimage[width=0.5\linewidth]{fig_sc2}\\
\end{figure}
\tiny{Bellec et al. HBM 2010. See Golland et al, 2008, Neuropsychologia, for more info on this resolution.}
}

\frame{
\frametitle{Group consensus clusters \MVAt(resolution 7)}
\begin{figure}
\pgfimage[width=\linewidth]{fig_sc7}\\
\end{figure}
\tiny{Bellec et al. HBM 2010. See Yeo et al., J Neurophysiol 2011, for more info on this resolution.}
}

\frame{
\frametitle{Sensorimotor network \MVAt(resolution 43)}
\begin{figure}
\pgfimage[width=\linewidth]{fig_rsn_sensorimotor_sc43}\\
\end{figure}
\tiny{Bellec et al. HBM 2010.}
}

\frame{
\frametitle{Sensorimotor network, subnetwork 1 (resolution 43)\MVAt(resolution 150)}
\begin{figure}
\pgfimage[width=\linewidth]{fig_CMA}\\
\end{figure}
\tiny{Bellec et al. HBM 2010.}
}

\section{Conclusions}

\frame{
\frametitle{Summary}
\begin{itemize}
 \item It is possible to identify resting-state networks (RSNs) at different levels and resolutions of analysis, using BASC.
 \item The estimation of the stability of RSNs is an important validation step.  
 \item Rather than identifying the ``correct'' resolution (an ill-defined problem in fMRI), MSTEPS seeks representative resolutions, to approximate accurately all stability matrices.
\end{itemize}
}

\frame{
\frametitle{Acknowledgements}
\begin{figure}
\pgfimage[width=\linewidth]{fig_acknowledgement}\\
\end{figure}
}

\end{document}